The DSPs read samples from the acqboard via SPORT0. We transfer in 11
16-bit samples, the 10 acquired from the acqboard as well as the
CMDSTS and CMDID bytes. Each DSP receives the same serial signal. Thus
each dsp receives ALL samples from the acqboard, and must know whether
to process the A set or the B set.

The acqboard's 8 MHz 8b/10b signal is deserialized and decoded by the
decoder module, and the frame disassembly module selects out the
individual 16-bit samples. The arrival of a complete, error-free frame
causes the assertion of NEWSAMPLE, which updates FrameSerialize's
internal registers.

FrameSerialize generates a single serial stream, which each associated
DSP receives.

\subsection{DSP Outbound commands}
A DSP creates a near-wirelevel packet to send to the acqboard. this is
to reduce the amount of hardware design necessary should the acqboard
change.

However, this means that each DSP can control all of the channels. a DSP must only send commands for its set of channels, A or B. 

To prevent commands from clobbering one another, the DSPio module services pending outbound requests in a round-robin fashion, and reads the cmdid from the pending request. The other DSP's request is not serviced until we receive the corresonding CMDID from the ACqboard. 

There is no mechanism to prevent, for example, one DSP trying to
switch the mode to 2 and another to 3, or DSPs simultaneously trying
to write the filter parameters.


\section{Framing}
Okay, screw all that above stuff -- we're going to see if we can use the ``multichannel'' mode to do exactly what we want. 

