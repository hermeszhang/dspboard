\documentclass{article}
\usepackage{fullpage}
\usepackage[pdftex]{graphicx}
\usepackage{somahw}


\begin{document}

The Soma DSPboard is the computational and signal processing core of
the Soma system. The DSPboard is responsible for decoding the optical
Acquisiton Board data stream, performing application-specific
filtering and processing on the incoming data, and passing the
resulting data onto the Data bus. The DSP Board also maintains tetrode
state, and provides an event bus interface for Acquisition board
parameters.

\section{Requirements}
For multiunit tetrode recording, the DSP Board must perform
user-programmable digital filtering on the raw 32 ksps 10-channel data
from the Acquisition board. Our signal processing requirements were as
follows:
    
\begin{itemize}
\item \textbf{Processing speed:} To perform 200-tap FIR filtering on each
  incoming sample would require 64 million multiply-accumulates per
  second -- we judged 200 coefficients to be the maximum conceivable
  length any user would need when filtering incoming data.
  
\item \textbf{Floating-point multiply-accumulate:} Fixed-point DSP requires
  careful consideration, especially in the IIR case, to avoid
  instability and overflow. This level of attention to filter
  coefficient selection is undesirable in any system where end-users
  are performing filter design, as it requires detailed understanding
  of the various stages of processing quantization. Thus, all DSP
  operations must use floatin-point arithmatic.
  
\item \textbf{Open/inexpensive development tools:} Any DSP we select needs to
  have inexpensive development tools, ideally ones that are Free
  Software. Since the bulk of DSP development will be performed in
  assembly, at the very least it would be ideal if a Free assembler
  existed.
  
\item \textbf{Ease of assembly:} Due to the limited anticipated volume
  production, any DSP considered needs to be avialble in a
  hand-solderable (non-BGA) package.
\end{itemize}

\section{DSP Board Hardware}
      
 The Analog Devices SHARC ADSP-21262 comes closest to meeting
the DSP criteria. The 200 MHz 32-bit floating point DSP has an SIMD
ALU allowing for up to 400 MMACS/second, and is available in a 144-pin
LQFP package. The development tools (VisualDSP++) are expensive but
available at a reduced price to academic institutions. To allow for
processing overhead, we use two DSPs, one per tetrode.  
      
A Xilinx Spartan-3 XCS200 FPGA is used to decode the Acquisition Board
optical data stream and pass it onto the DSPs, as well as providing
the necessary buffering and interface to the Data Bus and Event Bus.
      
\subsection{Interfaces}

\subsubsection{Data Bus}

\subsubsection{Event Bus}

\subsubsection{FPGA IO}
The following single-wire signals allow the FPGA to
signal the DSP about various sytem events. 


\begin{SignalTable}{FPGA/DSP IO}
  \signaldef{\signal{SAMPLE}}
  {A rising edge on this line tells the DSP that a
    new set of input samples is available from the
    FPGA. Occurs  every 31.25 microseconds.
  }
  
  \signaldef{\signal{EVENTS}}
  {A high indicates there are events waiting to be
    read in the FPGA's event bus queue. A successful
    read-out of all pending events by the DSP causes this
    line to go low. 
  }
        
  \signaldef{ \signal{TINC}}
  {Timer increment: a rising edge on this line
    signals the beginning of a new timestamp, every 100
    microseconds. 
  }
        
  \signaldef{\signal{TCLR}}
  {Clear timestamp counter. A high on this line
    coupled with a rising edge on TINC causes the timestamp
    counter to be reset to zero.
  }
\end{SignalTable}

\subsubsection{FPGA/DSP Memory IO}

 The ADSP supports several external serial interfaces, as well
as an asychronous 16-bit memory bus (the Parallel Port), the latter
being the only method with sufficient bandwidth for our application.
The ADSP-21262 automatically performs 16-to-32bit packing and
unpacking between the Parallel Port and its internal memory, and is
only able to access the Parallel Port via Direct Memory Access (DMA).


 The external memory space is partitioned into five
application-specific 12-bit regions. Several of the regions support a
"transaction complete" address that, when written or read, signals the
FPGA that the necessary writes or reads for that particular piece of
data are complete.  

\subsubsection{0x0nnn: Status and  Data  Samples}
This DSP-read-only region contains new sample data and status
information.
\input{statusanddata.memmap}


\subsubsection{0x2nnn: Output data buffer} 
 This region allows for the full write of a data frame destined
for the Data Bus, with a maximum length of 2048 bytes (1024
words).
\input{outputdata.memmap}

\subsubsection{0x4nnn: Event Output} 

A write of an Event Bus event to this region queues it in the FPGA for
transmission on the event bus.
\input{eventoutput.memmap}

\subsubsection{0x6nnn : Event Input}
The DSP can read new events from the Event Bus here. 
\input{eventinput.memmap}

\subsubsection{0x8nnn: Acqboard interface}
This region allows us to send commands to the acquisition board. 
\input{acqboardio.memmap}

\subimport{FPGA/}{FPGA} 

\end{document} 
