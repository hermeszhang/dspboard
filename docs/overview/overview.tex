
The Soma DSPboard is the computational and signal processing core of
the Soma system. The DSPboard is responsible for decoding the optical
Acquisiton Board data stream, performing application-specific
filtering and processing on the incoming data, and passing the
resulting data onto the Data bus. The DSP Board also maintains tetrode
state, and provides an event bus interface for Acquisition board
parameters.

\section{Requirements}
For multiunit tetrode recording, the DSP Board must perform
user-programmable digital filtering on the raw 32 ksps 10-channel data
from the Acquisition board. Our signal processing requirements were as
follows:
    
\begin{itemize}
\item \textbf{Processing speed:} To perform 200-tap FIR filtering on each
  incoming sample would require 64 million multiply-accumulates per
  second -- we judged 200 coefficients to be the maximum conceivable
  length any user would need when filtering incoming data.
  
\item \textbf{Floating-point multiply-accumulate:} Fixed-point DSP requires
  careful consideration, especially in the IIR case, to avoid
  instability and overflow. This level of attention to filter
  coefficient selection is undesirable in any system where end-users
  are performing filter design, as it requires detailed understanding
  of the various stages of processing quantization. Thus, all DSP
  operations must use floatin-point arithmatic.
  
\item \textbf{Open/inexpensive development tools:} Any DSP we select needs to
  have inexpensive development tools, ideally ones that are Free
  Software. Since the bulk of DSP development will be performed in
  assembly, at the very least it would be ideal if a Free assembler
  existed.
  
\item \textbf{Ease of assembly:} Due to the limited anticipated volume
  production, any DSP considered needs to be avialble in a
  hand-solderable (non-BGA) package.
\end{itemize}

\begin{figure}
\begin{center}
\includegraphics[scale=1.0]{board-overview.svg}
\end{center}
\caption{DSP Board Overview}
\end{figure}


 The Analog Devices SHARC ADSP-21262 comes closest to meeting
the DSP criteria. The 200 MHz 32-bit floating point DSP has an SIMD
ALU allowing for up to 400 MMACS/second, and is available in a 144-pin
LQFP package. The development tools (VisualDSP++) are expensive but
available at a reduced price to academic institutions. To allow for
processing overhead, we use two DSPs, one per tetrode.  
      
A Xilinx Spartan-3 XCS200 FPGA is used to decode the Acquisition Board
optical data stream and pass it onto the DSPs, as well as providing
the necessary buffering and interface to the Data Bus and Event Bus.

\section{DSP Board Signal Processing Overview}

\subsection{Spike channel Singal Chain}
\begin{figure}[h!]
\begin{center}
\includegraphics[scale=1.0]{spikesignal.svg}
\end{center}
\label{SpikeChain}
\caption{DSP board spike signal chain.}
\end{figure}

Samples are acquired via the optical interface and each tetrode's respective samples are passed to its DSP. Then (figure \ref{SpikeChain}): 

\begin{enumerate}
\item A received sample is placed into a fixed-length circular buffer. 
\item FIR or IIR filtering is performed on the sample. Each channel has its own filter. 
\item The sample is placed in a temporary output buffer
\item All channels are checked for spikes
\item When a spike is detected, the spike window is saved across all four channels, packetized, and trasmitted on the Data Bus. 
\end{enumerate}


Spike detection is achieved via simple thresholding; when the signal on a given challen crosses that channel's preset channel, four packets of length \textit{SPIKELEN} are formed (one per channel) with \textit{POSTTRIGLEN} samples after the trigger point (see figure \ref{threshold}). The 4-channel spike packet is then timestamped and sent to the event bus. 

\begin{figure}
\includegraphics[scale=2.0]{threshold.svg}
\caption{Overview of threshold and spike windowing}
\label{threshold}
\end{figure}

\subsection{Continuous channel operation}
\begin{figure}[h!]
\begin{center}
\includegraphics[scale=1.0]{contsignal.svg}
\end{center}
\label{ConChain}
\caption{DSP board continuous signal chain.}
\end{figure}

The continuous channel is almost identical to the spike channels, except that no spike detection is performed; rather the channel is optionally downsampled by a factor of DOWNSAMPLEFACTORSHOULDGOHERE. 
