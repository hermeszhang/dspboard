\section{Events}
The following is a general overview of how the DSP board interacts with the rest of the Soma system via the Event Bus. 

1. acqboard control
2. Configuration events
3. thresholding, spike windowing
4. buffer / filter configuration
5. timing 
6. Boot control

\subsection{Acquisition Board Control}
Any change of the analog acqboard state result in a tetrode state update event. This event is broadcast from the relevant tetrode to all devices. 

\begin{event}{CHANSTATE}
CMD: 0x45
SRC: TETN
DW0[3:0]: G1
DW0[7:4]: HWF1
DW1[3:0]: G2
DW1[7:4]: HWF2
DW2[3:0]: G3
DW2[7:4]: HWF3
DW3[3:0]: G4
DW3[7:4]: HWF4
DW4[3:0]: GC
DW4[7:4]: HWFC
DW4[9:8]: CS
\end{event}


Where \textbf{Gn} is the gain of channel \textbf{n} and \textbf{HWFn} is the high pass filter selection of channel \textbf{n}. For the continuous acquisition channel, \textbf{CS} is the currently muxed channel. 

[something about gain settings here] 

To explicitly query these settings, send 

\begin{event}{CHANQUERY}
CMD: 0x46
SRC: ANY
\end{event}

All tetrodes which receive such a query will broadcast a CHANSTATE. 

There is a single command to set the acqboard parameters with a fairly complex bitfield. The downside of a setting like this is that it will require many round-trip interactions with the acqboard if all bit fields are set. The upside is that it makes it easy for a single command to set all parameters at once. Remember that the acqstate event will only be broadcast once all of the relevant settings have been made. 


\begin{event}{SETACQ}
CMD: 0x47
SRC: TETN
DW0[0]: S
DW0[7:1]: G1
DW0[8]: S
DW0[12:9]: H1
DW1[0]:S
DW1[7:1]: G2
DW1[8]: S
DW1[12:9]: H2
DW2[0]:S
DW2[7:1]: G3
DW2[8]: S
DW2[12:9]: H3
DW3[0]:S
DW3[7:1]: G4
DW3[8]: S
DW3[12:9]: H4
DW4[0]:S
DW4[7:1]: GC
DW4[8]: S
DW4[12:9]: HC
DW4[13]: S
DW4[15:14]:C
\end{event}


The fields are:
\begin{itemize}
\item \textbf{Gn} Gain value to be set for channel \textbf{n} and whether or not this event is setting it (lsb \textbf{S})
\item \textbf{H1} Value to set the high pass filter for this channel to. \textbf{n}. 
\end{itemize}

As well as \textbf{S} to set the continuous channel input, and \textbf{C} as the channel number. 


\subsection{Configuration Events}
Most of the configuration parameters in the DSP board are set and queried via two events. This is to reduce the impact on the event space for events that are infrequently accessed. 

\subsubsection{Read Configuration Parameter}

\begin{event}{READPARAM}
CMD: 0x41
SRC: ANY
DW0[7:0]: TARGET
DW0[15:8]: ADDR
DW1: DW0[31:16]
DW2: DW0[15:0]
DW3: DW1[31:16]
DW4: DW1[15:0]
\end{event}


\subsubsection{Write Configuration Parameter}

\begin{event}{WRITEPARAM}
CMD: 0x42
SRC: ANY
DW0[7:0]: TARGET
DW0[15:8]: ADDR
DW1: DW0[31:16]
DW2: DW0[15:0]
DW3: DW1[31:16]
DW4: DW1[15:0]
\end{event}


\subsubsection{Configuration Parameters}

\paragraph{Write Filter}
\begin{dspcmd}{WRITEFILTER}
TARGET: 0x1n
ADDR: n
DW0[31:0]: 32-bit h[n]
DW1[31:0]: 32-bit h[n+1]
\end{dspcmd}


Writes the 32-bit filter coefficient. 

\paragraph{Filter Length}

\begin{dspcmd}{WRITEFILTERLENID}
TARGET: 0x20
ADDR: TGT
DW0[7:0]: LENGTH
DW1[15:0]: ID
\end{dspcmd}
Writes the filter length and the unique 16-bit filter ID. 

\paragraph{Spike Window Length}

\begin{dspcmd}{SPIKELEN}
TARGET: 0x23
DW0[7:0]: SPLEN
\end{dspcmd}

\paragraph{No Trigger Length Duration}

\begin{dspcmd}{NOTRIGGERLEN}
TARGET: 0x24
DW0[7:0]: NTLEN
\end{dspcmd}

\paragraph{Post-trigger duration}


\begin{dspcmd}{POSTTRIGLEN}
TARGET: 0x25
DW0[7:0]: PTLEN
\end{dspcmd}

Change continuous parameters

\begin{dspcmd}{DSRATIO}
TARGET: 0x26
DW0[3:0]: RATIO
\end{dspcmd}
The input for continuous data is downsampled by this ratio. A DSRATION of 2 means that every other input packet is discarded. 

\begin{dspcmd}{CFSIZE}
TARGET: 0x27
DW0[15:0]: FSIZE
\end{dspcmd}

Number of continuous samples per frame.

\begin{dspcmd}{THOLD}
TARGET: 0x28
DW0[31:0]: THRESHOLD
\end{dspcmd}

Number of continuous samples per frame.


\section{Data Bus Packets}
Packets on the data bus use a 16-bit word. All voltage data is transmitted in IEEE 32-bit single-precision float. 


\subsection{Spike Packets}
Spike events:
word 0: 0x01 to indicate spike event
word 1: source (DSP number)
word 2: ike length in 16-bit words, not including ID header
word 3: channel count (n)
word 4: LSBs of timestamp
word 5: MSBs of timestamp

Then n identical channels of the same length:
word 0: channel number
word 1: gain setting; actual gain
word 2: DSP Filter ID
word 3: hardware filter selection
word 4: lsbs of threshold
word 5: MSBs of threshold
words 6: LSBs of first sample
word 7 : MSBs of first sample

\subsection{Continuous Packets}
word 0: 0x02 to indicate a continuous event
word 1: length of this packet in 16-bit words
word 2: 
   7:0 source
  15:8 currently sampled channel
words 3-4: timestamp
5: gain setting
6: DSP filter ID
7: HW Filter
8: downsampling ration
14: data-forward, LSB first



