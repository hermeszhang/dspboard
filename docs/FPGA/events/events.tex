\section{DSP Events}

The interface between the DSPs and the Event Bus provides low-latency
buffering of inbound and outbound events. Outbound events are
multiplexed between the two DSPs and are placed on the bus during the
appropriate stage in an Event Bus cycle. Only events which target a
given DSP are placed in its buffer.

Here is where we'd put an overview of the different sub-components,
and describe the input/output dichotomy

All parts of the event bus face challenges in switching from the CLK
to the SYSCLK domain and vice-versa.

    <section  id="EventMux">
      <title>Event Bus Multiplexor </title>
      
      signals CEA and CEB for alternating DSPs A and B, alternating, via the ESEL signal
      
      Each DSP has a set of Address In (AI), Address Out (AO), Output Enable (OE), Data Out (DO), and Data In (DI) lines, as well as a signaling Clock Enable (CE). 

      The setting of MADDR : need some way to set wake-up signal, something something. 

      

    </section>
    
    <section id="Event Outputs">
       The event output subsystem uses the 16/32 bit dual-ported
      interface of the BlockSelect+ RAM in the Spartan-3 to enable
      simultaneous read out of the 16-bit data and 8 bit address
      during the event cycle, while providing an easy 16-bit interface
      to the Parallel Port for the DSP. 

      <section>
        <title>Writing an Event to the output buffer</title>
        <para> The input to the Output Event Buffer is complex due to
        the desire to preserve the 10-write-cycle length of an event
        write. A ROM maps the latched <signal>ADDRL[3:0]</signal> into
        the LSBs of <signal>ADDRA[9:0]</signal>, and the MSBs select the
        appropriate buffer location through the Event Counter In
        (<signal>ECNTIN[5:0]</signal>). 
        
        <para> The ROM mapping of addresses is necessary to have writes
        to locations 0x0, 0x1, and 0x2 map to the appropriate
        interleaved locations in the buffer for read-out.  Each
        <signal>WE</signal> assertion actually triggers
        <emphasis>two</emphasis> writes of the RAM buffer, with the ROM
        determining the necessary addresses and whether
        <signal>DINL[7:0]</signal> or <signal>DINL[15:8]</signal> maps
        to the low byte of <signal>DIA[15:0]</signal>. An event write is
        considered complete when the address 0x9 is written, thus
        incrementing <signal>ECNTIN[5:0]</signal>.
        
        
          <title> Example write transaction to event output buffer</title>
          <mediaobject>
            <timingobject name="output.write">
              <clock name="CLK">  | |  |  |  |  |  |  |  |  |  |  |  |  |  |  
              </clock>
              <signal name="WE">  L H  L  L  H  L  L  H  L  L  H  L  L  H  L
              </signal>
              <signal name="WEL"> L L  H  L  L  H  L  L  H  L  L  H  L  L  H
              </signal>
              <signal name="WELL">H L  L  H  L  L  H  L  L  H  L  L  H  L  L
              </signal>
              <bus name="ADDR[3:0]">
                <data>            0 0  0  0  1  1  1  2  2  2  3  3  3  4  4
                </data></bus>
              <bus name="ADDRL[3:0]">
                <data>            0 0  0  0  0  1  1  1  2  2  2  3  3  3  4                </data></bus>
              <signal name="AEN"> H H  H  H  H  H  H  H  H  H  H  L  L  L  L 
                </signal>
              <signal name="WEA"> L L  H  H  L  H  H  L  H  H  L  H  H  L  H
                </signal>
              <bus name="ADDRA[3:0]">
                <data>            1 0  0  2  0  4  6  4  8  10 8  1  1  1  3
                </data> </bus>
            </timingobject>
          </mediaobject>
        </figure>
                  
                
      </section>
      <section >
        <title> Reading an event from the output buffer</title>
        <para> The output side of the buffer is in the
        <signal>SYSCLK</signal> domain, and uses the 32-bit wide
        interface of the BlockSelect+ RAM to simultaneously read out
        the <signal>DATA</signal> and <signal>ADDR</signal> portions
        of the event. The 3-bit <signal>WCNT[2:0]</signal> counter
        sequences through the individual Event words, and the 6-bit
        <signal>ECNTOUT[5:0]</signal> counter selects the event from
        the buffer. These combine to form the address input
        <signal>ADDRB[8:0]</signal> to the RAM.  
   
        <para>A complicated interplay of signals is necessary to
        guarantee that <signal>EADDR[7:0]</signal> and
        <signal>EDATA[15:0]</signal> are valid as soon as possible
        following the <signal>SYSCLK</signal> edge that registers
        <signal>ECE</signal>. Thus, when not transmitting an Event,
        the counters point to the first word of the next event in the
        buffer. 

        <figure id="event.output.output">
          <title>Event Output </title>
          <timingobject name="event.output.output">
            <clock name="SYSCLK"> |   |   |   |   |   |   |   |   |   |   |   | 
            </clock>
            <signal name="ECE"  > L   L   L   H   L   L   L   L   L   L   L   L
            </signal>
            <signal name="EEVENT">L   L   L   H   L   L   L   L   L   H   L   L
            </signal>
            <bus name="ECNTINL">
              <data>            n+1 n+1 n+1 n+1 n+1 n+1 n+1 n+1 n+1 n+1 n+1 n+1
            </data></bus>
            <bus name="ECNTOUT">
              <data>              n   n   n   n   n   n   n   n   n   n n+1 n+1
            </data></bus>
            <signal name="ENOUT"> L   L   L   L   H   H   H   H   H   H  L  L 
            </signal>
            <signal name="WEN">   L   L   L   H   H   H   H   H   H   H  L  L 
            </signal>
            <bus name="WCNT">
              <data>              0   0   0   0   1   2   3   4   5   6  0  0
            </data></bus>
            <signal name="EEND">  L   L   L   L   L   L   L   L   L   H  L  L
            </signal>    
            <bus name="EDATA">
              <data>              0   0   0   0   D0  D1  D2  D3  D4  D5 0  0 
              </data> 
              <class>             2   2   2   2   1    1  1   1   1   1  2  2
            </class></bus>
            <bus name="EADDR">
              <data>              0   0   0   0   A0  A1  A2  A3  A4  A5 0  0
              </data>
              <class>             2   2   2   2   1   1    1   1   1   1 2  2
              </class>
            </bus>
          </timingobject>
        </figure>
        
        <para> An example event output can be seen in <xref
        linkend="event.output.output"/>. <signal>ECNTINL[5:0]</signal>
        is not equal to <signal>ECNTOUT</signal>, indicating the
        presence of a pending event. Note that
        <signal>ECNTINL[5:0]</signal> would only be incremented
        following the complete write of an event, thus there are no
        potential timing conflicts. The simultaneous assertion of
        <signal>EEVENT</signal> and <signal>ECE</signal> with events
        in the buffer leads to the setting of
        <signal>ENOUT</signal>. This enables the output muxes to drive
        the output data and address lines with the real values from
        the event buffer. 

        <para> <signal>WCNT[2:0]</signal> is incremented while
        <signal>ENOUT</signal> is high <emphasis>and</emphasis> when
        <signal>EEVENT</signal> and <signal>ECE</signal> are
        simultaneously asserted. Since the BlockSelect+ RAM is
        synchronous (and thus the data output will follow the input
        address by a clock signal) <signal>WCNT[2:0]</signal> must
        begin incrementing early.

        <para><signal>EEND</signal> goes high when
        <signal>WCNT[2:0]</signal> equals six, indicating the
        completion of an event write. This resets
        <signal>ENOUT</signal> and thus resets
        <signal>WCNT[2:0]</signal> while incrementing
        <signal>ECNTOUT[5:0]</signal>. 
      </section>
    
    </section>

    <section>
      <title>Event Reader Stage</title>
      <para> Here's where we read the events off the Event Bus and
      potentially store them if they are for us, that is, if the
      relevant ADDR bit is set.  
      
      <para>This is a challenge because we need to be able to read an
      event and store it in our event buffer, <emphasis>even
      if</emphasis> we get two events in a row. Thus, we use lots of
      pipelining. 

      <para> To determine if an event is for us, we latch in each
      <signal>EADDR[7:0]</signal> byte, and use
      <signal>MADDR[2:0]</signal> to select the correct bit onto
      <signal>ADDRSEL</signal>. The counter <signal>CNT</signal> is
      reset to zero in the cycle following <signal>EVENTL</signal>
      going high. <signal>SMINE</signal> is set when
      <signal>CNT</signal> is equal to
      <signal>MADDR[5:3]</signal>. <emphasis><signal>SMINE</signal>'s
      set overrides its reset, which is cleared on each cycle -- this
      is necessary to allow setting when the relevant EADDR byte is at
      CNT=0. </emphasis>

      <para>Each of <signal>D0[15:0]</signal> through
      <signal>D5[15:0]</signal> is latched during the appropriate
      state in the Event, and the assertion of <signal>LMINE</signal>
      latches the corresponding values to the registered
      <signal>D0L[15:0]</signal> through <signal>D5L[15:0]</signal>,
      respectively. These are read asynchronously via the outputs when
      <signal>MINE</signal> is eventually set. 

      <figure>
        <title> Example timing</title>
        
        <mediaobject>

          <xi:include  href="event.timing.xml" 
                       xmlns:xi="http://www.w3.org/2003/XInclude">
            <xi:fallback>
              <para><emphasis>FIXME:  MISSING XINCLUDE CONTENT
              </emphasis>
            </xi:fallback>
          </xi:include>
        </mediaobject>
      </figure>
    </section> 

    <section>
      <title> Event Inputs</title>
      <para> The transfer of the Events from the Event Bus to the DSP
      is relatively simple compared to the other portions of the Event
      Subsystem. As the Event Input system is in the
      <signal>SYSCLK</signal> domain, we wait for the falling edge of
      the <signal>EIN</signal> via <signal>EINDELTA</signal>. Note
      that timing here is critical: A full event cycle in the
      <signal>SYSCLK</signal> domain takes at most 19 ticks in the
      <signal>CLK</signal> domain. 
      
      <para> The FSM waits for an assertion of
      <signal>EINDELTA</signal> and then checks for
      <signal>MINE</signal> to be high, indicating that the current
      event is targeted for this DSP. Then,
      <signal>EDOUT[7:0]</signal> is read to determine the CMD of the
      event. A small subset of the potential events, related to
      resetting and booting, are handled by the FPGA. All others are
      placed in a buffer for the DSP. 

      <section>
        <title>Mode Set</title>
        <para> When the Event's CMD is 0x01, we transition to
        <state>MODEEN</state> and set the <signal>MODE</signal> bit
        appropriately. Note that <signal>MODE</signal> can only be
        reset by the external <signal>MODERST</signal> line. 
      </section>
      
      <section>
        <title>DSP Reset</title>
        <para> The <signal>DSPRESET</signal> signal is determined in a
        similar manner, as the complement to the first data word of
        the Reset Event
      </section>
      
      <section>
        <title> Boot RAM writes</title>
        <para> <signal>RAINCNT[9:0]</signal> is first loaded with the
        target address of the location in the boot RAM. The following
        four words are written to that and the three subsequent
        addresses via <signal>RDIN[15:0]</signal>by incrementing
        <signal>RAINCNT[9:0]</signal> three times. 
      </section>

      <section>
        <title>Events targeted for the DSP</title>
        <para> All events not handled internally by the FPGA are
        placed in a BlockSelect+ RAM circular buffer for later reading
        by the DSP. The first word of each event is
        <emphasis>always</emphasis> written to the current address
        pointed to by <signal>AEIN[9:0]</signal>, regardless of
        <signal>MINE</signal> status. This should never be a problem
        as <signal>AEIN[9:0]</signal> should point to the next empty
        buffer location at the start of each event read cycle. The
        next five words are read and written sequentially into the
        buffer, with the low bits of the address
        (<signal>AEIN[2:0]</signal>) being controlled by the FSM. The
        last stat ein the FSM incrmeents <signal>AEIN[9:3]</signal>,
        positioning the system to write the next incoming
        event. 

        <para> Readout is similar to other places in the FPGA, with
        <signal>NEWEVENTS</signal> signalling the presence of a
        non-empty buffer. <signal>ADDR[2:0]</signal> synchronously
        reads the relevant words from the buffer, with a read of
        location 0x6 signalling the completion of a read and thus
        incrementing <signal>ADSPIN[9:3]</signal>.  
      </section>
    </section>
  </section>
  <section>
    <title> Present concerns</title>
    <para> 1. For the event input stage, what kind of latch is SMINE? how is it possible for things to work when the MYADDR[5:3]=111, i.e.  in the last byte of the address? 
  </section>


</article>
        
