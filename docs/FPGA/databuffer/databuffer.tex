
\subsection{Data Buffer}

\begin{figure}[h]
  \includegraphics[scale=0.8]{databuffer.pdf}
  \caption{Data and Boot Buffer.}
  \label{databuffer}
\end{figure}

The Data Buffer (Figure \ref{databuffer} is responsible for
triple-buffering data packets from the DSP to the Data Bus, and
providing a temporary storage location for the DSP's 256-op loader
kernel prior to boot.
    
Triple-buffering of data is accomplished using 3 blocks of
BlockSelect+ RAM in the Spartan-3 FPGA, with a total of 2048 bytes per
block. Thus the maximum size of a data packet is 2048 bytes, including
header. Packets are guaranteed to be read out in full, and overwrites
result in the overwriting packet being lost -- thus overwrites lead to
lost, not corrupt, packets.



\begin{figure}
  \includegraphics[scale=0.8]{DataMux.pdf}
\end{figure}
        
\subsubsection{Data Buffer Input}
        
      
The \signal{BUFWE}, \bus{BUFDIN}{15:0}, and \bus{BUFADDRIN}{9:0} write
data bytes into the input buffers when \signal{MODE} is low, i.e. the
system is not in boot mode. Register \bus{BUFLENIN}{9:0} is latched
when the address corresponding to the Length field in the data packet
is written, thus setting \bus{BUFLENIN}{9:0} to the total length of
the data packet. The last write of the packet occurs when
\bus{BUFADDRIN}{9:0} is equal to one less than \bus{BUFLENIN}{9:0},
and results in \signal{BUFWDONE} being asserted.

The input state machine \signal{incs} (Figure \ref{InputFSM})is responsible for selecting the
currently written buffer. When \signal{incs} is in state A, B, or C,
the corresponding \signal{BUFWEA}, \signal{BUFWEB}, or \signal{BUFWEC}
is enabled, respectively. When \signal{ BUFWDONE} is asserted, the
corresponding FULL{A,B,C} SR flip-flop is set, and the FSM moves into
a WAIT{A,B,C} state.

      
\begin{figure}
  \includegraphics[scale=0.8]{input.FSM.pdf}
  \caption{Data Buffer input FSM}
  \label{InputFSM}
\end{figure}

The next packet can only be written when the next buffer is empty, and
the system has moved out of the corresponding WAIT state. While
\signal{incs} is in a wait state, the \bus{BUFLENIN}{9:0} register is
reset to the highest-possible value to prevent accidental triggering.
      
\subsubsection{Data Buffer Output}
      
The outputs of the three buffers are multiplexed into the
\bus{BUFDOUT}{15:0} signal, which is optimized for interfacing with
the system Data Bus. Similar to the input stage, the output reads the
Length word in the data header to determine total packet length.

\begin{figure}[h]
  \includegraphics[scale=0.8]{output.FSM.pdf}
  \caption{Data Buffer output FSM.}
  \label{OutputFSM}
\end{figure}
The output FSM \signal{outcs} (Figure \ref{OutputFSM}) consists of
three similar phases, one for each buffer. The system sits in a WAIT
state until the corresponding \signal{FULL{A,B,C}} is asserted
coincident with the \signal{NEXTOUT}, advancing to the corresponding
A, B, or C. The resulting high on \signal{BUFCNTEN} causes the
\bus{BUFCNT}{9:0} counter to advance on each output word.
\bus{BUFLENOUT}{9:0} latches the appropriate length word.
\signal{BUFCNTEN} is delayed by one tick and output as
\signal{BUFACKOUT} to the \signal{DACK} on the Data Bus.

Upon \bus{BUFLENOUT}{9:0} equaling \signal{BUFCNT}{9:0}, the writing
of the packet is complete, and the transition to the DONE state clears
the corresponding \signal{FULL{A,B,C}}. The FSM then waits for the
deassertion of \signal{NEXTOUT} before advancing to the wait state for
the next buffer.
   
\subsubsection{Boot Buffer}

BlockSelect+ RAM Buffer A is also used to store the loader kernel for
the DSP. This allows us to buffer the loader kernel into the FPGA
before booting the DSP, which expects to rapidly read the kernel from
RAM. The EEPROM-based boot mode of the DSP uses the Parallel port in
8-bit mode.

When the DSP is in boot mode (\signal{MODE} is high) and the DSP is
held in a reset state (\signal{DSPRESET}=1), \signal{RWE},
\bus{RAIN}{9:0}, and \bus{RDIN}{15:0} can write to the FPGA. The
deassertion of \signal{DSPRESET} switches the buffer address input to
the 11-bit \bus{RAOUT}{10:0}, with the 10 MSBs selecting the data word
from the buffer and the LSB controlling which byte drives
\bus{RDOUT}{7:0}

